\documentclass{jsarticle}
\usepackage[margin=2.54truecm]{geometry}

\title{セルオートマトン}
\author{ctes091x}

\begin{document}
\maketitle

\section{はじめに}
はじめまして\footnote{ここでは世間一般で通用する挨拶を用いましたが、マイコン部では「グッドモルニッヒ!」という挨拶が主に用いられています。}。本郷高校マイコン部のctes091xです。本稿では、簡単な規則から豊かな結果を得ることのできる数理モデルのひとつであるセルオートマトンについて解説します。

\emph{セルオートマトン}は、セルによって構成された数理モデルです。\emph{セル}は、有限の数の状態を持ち、\emph{近傍}のセルと自身が現在とっている状態をもとにシンプルな規則に従って変化させていくような、セルオートマトンの最小単位を指します。全てのセルは同じ規則を共有し、状態は同時に更新されます\footnote{これに当てはまらないものとして\emph{確率的セルオートマトン}や\emph{非同期セルオートマトン}なども存在しますが、本稿では扱いません。}。

\section{例}
セルオートマトンの簡単な例として、人が細い通路を1列になって進む状況の再現を考えてみます。

初めに、細い通路を表現するものとして、多数のセルを1列に並べたものを用意します。各セルは0と1の2状態をとり、状態0をそのセルに人がいない状態、1を人がいる状態に対応させることとします。各セルの状態は左右のセルと自身が現在とっている状態のみに基づいて変化するものとします。

次に、セルの状態遷移の規則を考えます。ここでは人が左から右へ進んでいくこととします。細い通路を進む際、前に十分なスペースが無ければ前に進むことはできません。よって、人は右のセルが空いているときのみ右に移動することができるとするのが妥当でしょう。個々のセルはその両隣にまでしか影響を及ぼすことはできないので、1単位時間で人は最大で1セルだけ右に進むことができます。

% \begin{table}[hbtp]
%     \caption{細い通路を進む人を表すセルオートマトンの時間発展の例}
%     \label{table:example-of-rule184}
%     \centering
%     \begin{tabular}[|c|cccccc|]
%         \hline
%         時間 & セルの状態 
%     \end{tabular}
% \end{table}

\end{document}